\par The problem, motivated from a mathematical perspective:
\begin{itemize}
    \item Dimensionality reduction 
    \item Spectral clustering
    \item Random walks on graphs/groups (equivalent reformulations of the same problem)
    \item From the frequency domain (spectrum) to the physical domain (geometry)
    \item Non-abelian harmonic analysis
\end{itemize}
\par More on the connections between neuroscience and mathematics, based on Citti and Sarti's collection, \textit{Neuromathematics of Vision} \cite{citti_neuromathematics_2014}:
\begin{itemize}
    \item sub-Riemannian geometry is important to model long-range horizontal connections
    \item harmonic analysis in non commutative groups is fundamental to understanding the pinwheels structure 
    \item non-linear dimensionality reduction is at the base of many neural morphologies and possibly of the emergence of perceptual units
\end{itemize}

\section{Spectral Graph Theory}

\subsection{Graph Laplacian}
\begin{itemize}
    \item Discretization of the Laplacian operator
    
  In one dimension, the Laplacian is the second derivative, which can be approximated with the following discretization,
\begin{align}
    \frac{d^2}{dx^2}r(x_i) \approx \frac{r(x_{i-1}) + r(x_{i+1}) - 2 r(x_i)}{(\Delta x)^2},
\end{align}
where $\Delta x = x_j - x_{j-1}$ for all $j$.

This gives us the following matrix approximation for the Laplacian operator:
\begin{equation}
L_n = \frac{1}{(\Delta x)^2}
\begin{pmatrix}
    -1  &1  &   &   &   &   &\\
    1   &-2 &1  &   &   &   &\\
        &1  &-2 &1  &   &   &\\
        &   &   &\ddots & & &\\
        &   &   &   &1  &-2 &1\\
        &   &   &   &   &1  &-1
\end{pmatrix},
\end{equation}
where we use the Neumann boundary conditions to choose the top left and bottom right entries. The Neumann boundary condition in this case is $r'(x_{\text{boundary}}) = 0$.

\item Graph Laplacian and random walks

\end{itemize}

\subsection{Inverse problems in spectral geometry}
References for this section: \cite{lablee_spectral_2015}, \cite{kac_can_1966}
\textbf{Main idea: connections between the geometry of the manifold and the spectrum of a linear unbounded operator on that manifold. }

Given a compact Riemannian manifold $(M,g)$, we can associate to it a linear unbounded operator $-\Delta_g$. We denote the spectrum of $-\Delta_g$ by 
$$Spec(M,g) = (\lambda_k(M))_k.$$

Equivalently, $\forall k \geq 0$, there exists a non-trivial eigenfunction $u_k$ on $M$  such that 
$$-\Delta_g u_k = \lambda_k(M) u_k.$$

\begin{itemize}
    \item Direct problems: 
    
    Given a compact Riemannian manifold $(M,g)$, can we compute the spectrum $Spec(M,g)$? And can we find properties on the spectrum $Spec(M,g)$?
    \item Inverse problems: 
    \begin{enumerate}
        \item Does the spectrum $Spec(M,g)$ determines the geometry of the manifold $(M,g)$? 
        \begin{itemize}
            \item the dimension of $(M,g)$
            \item the volume of $(M,g)$
            \item the integral of the scalar curvature $Scal_g$ over $(M,g)$.
        \end{itemize}
        
        \item What sequences of real numbers can be spectra of a compact manifold?
        
        \item The spectrum of the manifold determines its length spectrum. 
        
        (The length spectrum of a compact Riemannian manifold $(M,g)$ is the set of lengths of closed geodesics on $(M,g)$ counted with multiplicities.)
        
        \item If two Riemannian manifolds $(M,g)$ and $(M^\prime,g^\prime)$ are isospectral, are they isometric?
    \end{enumerate}
    \end{itemize}
    