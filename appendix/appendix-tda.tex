\chapter{Persistent Homology Theory} 
\label{appendix-tda} 

In this appendix, we provide the full details on the theoretical background of persistent homology.
The main reference for the following exposition is \cite{carlsson_topology_2009}
% ) and (\cite{zomorodian_computing_nodate}).

\section{Motivation}

First of all, there are four major advantages for using topological methods to analyze high-dimensional point clouds.

\begin{enumerate}
	\item Topology provides qualitative information which is required for data analysis.
	
	\item Metrics are not theoretically justified. Compared to straightforward geometric methods, Topology is less sensitive to the actual choice of metrics. 
	
	\item Studying geometric objects using Topology does not depend on the coordinates. 
	
	\item Functoriality. This is the most important advantage.
	
	\begin{defn}[Functoriality]
	
		For any topological space X, abelian group $A$, and integer $k\geq 0,$ there is assigned a group $H_k(X,A).$ For any $A$ and $k$, and any continuous map $f: X \to Y,$ there is an induced homomorphism $H_k(f,A): H_k(X,A) \to H_k(Y,A).$ Then \underline{functoriality} refers to the following conditions:
		\begin{itemize}
			\item  $H_k(f\circ g,A): H_k(f,A) \circ H_k(g,A)$
			\item  $H_k(Id_{X};A) = Id_{H_k(X,A)}.$
		\end{itemize}
	\end{defn}
	Functoriality addresses the ambiguities in statistical clustering methods -- in particular the arbitrariness of various threshold choices. We now illustrate how exactly functoriality could be used in questions related to clustering. 
	
	Let $X$ be the full data set and $X_1,X_2$ are the subsamples from the data set. If the set of clusterings $C(X_1), C(X_2), C(X_1 \cup X_2)$ correspond well, then we can conclude that the subsample clusterings correspond to clusterings in the full data set $X$. 
\end{enumerate}

\section{Homotopy} \label{sec:homotopy}

\begin{defn}[Homotopy]

	\underline{Homotopy} is a family of maps $f_t: X \to Y$ where $t \in I$ such that $F: X\times I \to Y$ defined by $F(x,t) \mapsto f_t(x)$ is continuous.
	
	Two maps $f_0, f_1$ are \underline{homotopic} if there exists a homotopy $f_t$ between $f_0$ and $f_1$.  
\end{defn}

\begin{defn}[Homotopy equivalence]
A map $f:X \to Y$ is a \underline{homotopy equivalence} if there is a map $g: Y \to X$ such that 
\begin{itemize}
	\item $f\circ g$ is homotopic to the identity map on $Y$, and
	\item $g\circ f$ is homotopic to $f$.
\end{itemize}

Two spaces $X,Y$ are\textit{ homotopy equivalent} if there exists a homotopy equivalence $f: X \to Y. $
\end{defn}

\begin{thm}
	If $f$ and $g$ are homotopic, then $H_k(f,A) = H_k(g,A).$ 
	
	If $X$ and $Y$ are homotopy equivalent, then $H_k(X,A) \cong H_k(Y,A) $.
	
	Note that if two spaces are homotopy equivalent, then all their Betti numbers (defined in subsequent section) are equal.
\end{thm}
\begin{defn}[Topological space]
    Associated to a simplicial complex $(V, \triangle)$ is a topological space $|(V,\triangle)|$. $|(V,\triangle)|$ may be defined using a bijection $\phi : V \to \{1, 2, \dots,N\}$ as the subspace of $\mathbb{R}^N$ given by the union 
	$$\bigcup_{\sigma \in \triangle} c(\sigma),$$
	where $c(\sigma)$ is the convex hull of the set $\{e_{\phi(s)}\}_{s\in \sigma}$, where $e_i$ denotes the $i$-th standard basis vector.

	 We often use abstract simplicial complexes to approximate topological spaces. For simplicial complexes the homology can be computed using only the linear algebra of finitely generated $\mathbb{Z}$-modules. In particular, for simplicial complexes, homology is algorithmically computable.
\end{defn}	
	
\begin{defn}[Nerve]
	Let $X$ be a topological space, and let $\mathcal{U} = \{U_\alpha\}_{\alpha\in A}$ be any covering of $X$. 
	
	The \underline{nerve} of $\mathcal{U}$, denoted by $N(\mathcal{U})$, will be the abstract simplicial complex with vertex set $A$, and where a family $\{\alpha_0, \dots, \alpha_k\}$ spans a $k$-simplex if and only if $U_{\alpha_0}\cap U_{\alpha_1} \cap \cdots \cap U_{\alpha_k} \neq \emptyset.$
\end{defn}

One reason that this construction is very useful in the following ``Nerve Theorem." This theorem gives the criteria for $N(\mathcal{U})$ to be homotopy equivalent to the underlying topological space $X$.

\begin{thm}[Nerve Theorem]
Suppose that $X$ and $U$ are as above, and suppose that the covering consists of open sets and is numerable. Suppose further that for all $\emptyset \subseteq A,$ we have that $\bigcap_{s\in S} U_s$ is either contractible or empty. Then $N(\mathcal{U})$ is homotopy equivalent to $X$.
\end{thm}
The Nerve Theorem is very important in TDA since it provides a way to encode the topology of continuous spaces into abstract combinatorial structures that are more useful for designing computationally efficient data structures and algorithms.

\section{Simplicial Complexes}

\begin{defn}[Simplicial complex]

	An \underline{abstract simplicial complex} is a pair $(V, \triangle)$, where $V$ is a finite set, and $\triangle$ is a family of non-empty subsets of $V$ such that 
	$$\tau \in \triangle \text{ and }\sigma \subseteq \tau \implies \sigma \in \triangle.$$
	
	$\tau \in \triangle $ is face of $\triangle$. The dimension of a face $\tau$ is $|\tau| - 1.$
	
	(Intuition) A simplicial complex $\triangle$ in $\mathbb{R}^n$ is a collection of simplices in $\mathbb{R}^n$ such that
	\begin{enumerate}
	    \item Every face of a simplex of $\triangle$ is in $\triangle$.
	    \item The intersection of any two simplicies of $\triangle$ is a face of each. 
	\end{enumerate}
\end{defn}

\begin{defn}[f-vector]

    For each $n \geq -1,$ let $f_n = f_n(\triangle)$ be the number of faces of $\triangle$ of dimension $n$.
    
    The f-vector of $\triangle$ is the vector $(f_{-1}, f_0, f_1, \dots, f_d)$ where $d$ is the dimension of $\triangle$. 
    
    \begin{eg}
    Suppose the family of sets 
    $$\emptyset, \{a\},\{b\},\{c\}, \{d\}, \{a,b\}, \{a,c\}, \{b,c\}, \{b,d\}, \{c,d\}, \{a,b,c\}$$ form a simplicial complex $E_1$. Then the f-vector of $E_1$ is $(1,4,5,1)$.
    \end{eg}
\end{defn}

\begin{defn}[Oriented complex]

(Intuition) An \underline{oriented simplex} is a simplex $\sigma$ together with an orientation of $\sigma$. If $\{a_0, a_1, \dots, a_p\}$ spans a $p$-complex $\sigma$, then we denote the oriented simplex with $[a_0, a_1, \dots, a_p]$.

(Formal) Let $\mathbb{F}$ be a commutative ring, $\triangle$ be a simplicial complex. For each $n \geq -1,$ we form a free $\mathbb{F}$-module $\Chain_n(\triangle; \mathbb{F})$ with a basis indexed by the $n$-dimensional faces of $\triangle$. For each n-dimensional face $a_0 a_1\cdots a_n$, we have a basis element $\vec{e}_{a_0,a_1,\cdots, a_n}$. 

We refer to the basis element $\vec{e}_{a_0,a_1,\cdots, a_n}$ as an \underline{oriented simplex}. The concept of an oriented simplex is analogous to the idea of a unit vector (which gives the direction and has unit length).
\end{defn}

\begin{defn}[Chain group of degree $n$]
    We refer to the free $\mathbb{F}$-module $\Chain_n(\triangle; \mathbb{F})$ in the previous definition as the chain group of degree $n$. The rank of  $\Chain_n(\triangle; \mathbb{F})$ is the $n$-th value $f_n(\triangle)$ in the f-vector of $\triangle$.

\begin{eg}
For $E_1 = \{\emptyset, \{a\},\{b\},\{c\}, \{d\}, \{a,b\}, \{a,c\}, \{b,c\}, \{b,d\}, \{c,d\}, \{a,b,c\}\}$, we have the following chain groups of degrees $-1,0,1,2$, respectively:
\begin{itemize}
    \item $\Chain_{-1}(E_1) = \{\lambda e_{\emptyset}: \lambda \in \mathbb{F}\} \cong \mathbb{F}.$
    \item $\Chain_{0}(E_1) = \{\lambda_{a} e_{a} + \lambda_{b} e_{b}  + \lambda_{c} e_{c} + \lambda_{d} e_{d}: \lambda_{a}, \lambda_{b}, \lambda_{c}, \lambda_{d} \in \mathbb{F}\} \cong \mathbb{F}^4.$
    \item $\Chain_{1}(E_1) = \{\lambda_{a b} e_{a,b} + \cdots  + \lambda_{c d} e_{c,d} : \lambda_{a b}, \dots,  \lambda_{c d} \in \mathbb{F}\} \cong \mathbb{F}^5.$
    \item $\Chain_{2}(E_1) = \{\lambda e_{a,b,c}: \lambda \in \mathbb{F}\} \cong \mathbb{F}.$
\end{itemize}
\end{eg}
\end{defn}

\begin{defn}[Simplicial chain complex]
The \underline{simplicial chain complex} of a simplicial comlex $\triangle$, denoted with $C(\triangle)$, is defined as:

\begin{tikzcd}[cells={nodes={minimum height=2em}}]
\cdots\arrow[r,"\partial_{n+2}"] & 
\Chain_{n+1}(\triangle) \arrow[r,"\partial_{n+1}"] & 
\Chain_{n}(\triangle) \arrow[r,"\partial_{n}"] & 
\Chain_{n-1}(\triangle) \arrow[r,"\partial_{n - 1}"] &
\cdots
\end{tikzcd}

\begin{eg}
The simplicial chain complex of $E_1$,  $C(E_1)$ is 

\begin{tikzcd}[cells={nodes={minimum height=2em}}]
0\arrow[r] & 
\Chain_{2}(E_1) \arrow[r,"\partial_{2}"] \arrow[d, "\cong"] & 
\Chain_{1}(E_1) \arrow[r,"\partial_{1}"] \arrow[d, "\cong"] & 
\Chain_{0}(E_1) \arrow[r,"\partial_{0}"] \arrow[d, "\cong"] &
\Chain_{-1}(E_1) \arrow[r] \arrow[d, "\cong"] &
0\\
0\arrow[r] & \mathbb{F} \arrow[r,"\partial_{2}"] & \mathbb{F}^5  \arrow[r,"\partial_{1}"] & \mathbb{F}^4 \arrow[r,"\partial_{0}"] &
\mathbb{F}  \arrow[r] &
0
\end{tikzcd}
Recall that $E_1$ has f-vector $(1,4,5,1)$.
\end{eg}
\end{defn}
\begin{prop}[The double boundary condition]
\label{double-boundary}
Boundary maps satisfy the double boundary condition, that is, $\partial_n \circ \partial_{n+1} = 0 \quad \forall n.$
\end{prop}
\begin{defn}[Simplicial homology of degree $k$]
\label{kth-homology-group}
    (Intuition) The simplicial homology gives an algebraic measure on the amount of cycles that are not the boundaries. 

    (Formal) Define he $\mathbb{F}$-module $Z_k(\triangle; \mathbb{F})$ of cycles and the $\FF$-module of he boundary group $B_k(\triangle;\FF)$ by the following formulas:
    \begin{itemize}
        \item $Z_k(\triangle; \mathbb{F}) = ker(\partial_k) = \{Z \in \Chain_k(\triangle; \FF): \partial_k(Z) = 0\}$ 
        \item $B_k(\triangle;\FF) = im(\partial_{k+1}) = \{Z \in \Chain_k(\triangle; \FF): \partial_{k+1}(x), \quad x \in \Chain_{k+1}(\triangle; \FF)\}$ 
    \end{itemize}
    
    By \ref{double-boundary}, $B_k(\triangle;\FF)$ is a submodule of $Z_k(\triangle;\FF)$, i.e., $B_k(\triangle;\FF)\subseteq Z_k(\triangle;\FF) \subseteq C_k(\triangle;\FF)$. 
    
    We define the \underline{simplicial homology in degree $k$ of $\triangle$} to be the quotient group
    \begin{align}
        \Hom_k(\triangle;\FF) &= ker(\partial_k) / im(\partial_{k+1})\\
        &= Z_k(\triangle;\FF) / B_k(\triangle;\FF)
    \end{align}
\end{defn}

\begin{defn}[$k$-th Betti number]
\label{kth-betti}
    The $k$-th Betti number of the simplicial complex $\triangle$ is
    \begin{align}
        \beta_k(\triangle;\FF) &\coloneqq \dim \Hom_k(\triangle;\FF) \\
        &= \dim ker(\partial_k) - \dim im(\partial_{k + 1}).
    \end{align}
    
    \begin{itemize}
        \item elements in $ker(\partial_k)$ are called $k$-cycles.
        \item elements in $im(\partial_k)$ are called $k$-boundaries.
        \item $k$-cycles that are not boundaries represent $k$-holes, which means that the $k$-th Betti number $\beta_k(\triangle; \FF)$ is the number of $k$-holes. 
    \end{itemize}
\end{defn} 

\begin{defn}[Homology as ``holes"]
In certain well-behaved cases, homology can be interpreted as an algebraic measure on the amount of ``holes" in a simplicial complex $\triangle$.

Formally, given $X\subseteq \RR^n$, a ``hole" of $X$ is a bounded connected component of $\RR^n\ X$. For a given $k\geq 0,$ suppose the $(k+1)$-skeleton $\triangle^{(k+1)}$ of $\triangle$ has a geometric realization $X \subseteq \RR^n$. Then $\Hom_k(\triangle;\FF)$ is a free $\FF$-module of rank equal to the number of holes of $X$. 

Homology associates a vector space $H_k(X)$ to a topological space $X$ for each $k\in \NN$.
\begin{itemize}
    \item $H_0(X)$ is the number of path components in $X$.
    \item $H_1(X)$ is the number of holes in X.
    \item $H_2(X)$ is the number of voids in X.
\end{itemize}
\begin{rmk}
Note, however, that in the general metric spaces, the interpretation that homology measures the amount of ``holes" does not hold.
\end{rmk}
\end{defn}

\section{Persistent homology}

The main idea of \textit{persistence} is that instead of selecting a fixed value of the threshold $\epsilon$, we would like to obtain a useful summary of the homological information for all the different values of $\epsilon$ at once. As $\epsilon$ increases, we add simplicies to the complexes and detect which features ``persist."

The main advantages of persistent homology are threefold: 
\begin{enumerate}
    \item It is based on algebraic topology.
    \item It is computable via linear algebra.
    \item It is robust with regards to perturbations in the input data. 
\end{enumerate}

\begin{defn}[Filtered simplicial complex]
\label{filtered}
A subcomplex of $\triangle$ is a subset $\triangle^i \subseteq \triangle$ that is also a simplicial complex. 
Let $\triangle$ be a finite simplicial complex and let $\triangle^1 \subset \triangle^2 \subset \cdots \subset \triangle^m = \triangle$ be a finite sequence of nested subcomplexes of $\triangle$. The simplicial complex $\triangle$ with such a sequence of subcomplexes, $\emptyset \subseteq \triangle^1 \subseteq \triangle^2 \subseteq \cdots \subseteq \triangle^m = \triangle$, is called \underline{filtered simplicial complex}.

\end{defn}

\begin{defn}[$p$-persistent $k$-th homology group]
Given a filtered complex, for the $i$-th subcomplex $\triangle^i$ we can compute the associated 
\begin{itemize}
    \item the boundary maps $\partial_k^i$ for all dimensions $k$.
    \item boundary matrices $M_k^i$ for all dimensions $k$.
    \item groups $C_k^i, Z_k^i$ (cycle group), $B_k^i$ (boundary group), and $H_k^i$ (homology group).
\end{itemize}
Then the \underline{$p$-persistent $k$-th homology group} $H_k^{i,p}$ of $\triangle^i$ is 
\begin{itemize}
    \item $H_k^{i,p} = Z_k^i / (B_k^{i+p}\cap Z_k^i)$
    \item (equivalently) $H_k{i,p} \cong Im(\eta_k^{i,p})$, where $\eta_k^{i,p}$ is a bijection $\eta_k^{i,p}: H_k^i \to H_k^{i+p}$ that maps a homology class into another homology class containing it.   
\end{itemize}
Note that this is simply the definition of homology group of degree $k$ in Definition \ref{kth-homology-group} with the additional notion of persistence.
\end{defn}

\begin{defn}[$p$-persistent $k$-th homology group]
We now have the persistent version of \ref{kth-betti}:

The $p$-persistent $k$-th Betti number of $\triangle^i$ is $\beta_k{i p} = $ the rank of the free subgroup of $H_k^{i,p}$.
\end{defn}


\begin{defn}[Persistence complex]
A \underline{persistence complex} $\mathscr{C}$ is a family of chain complexes $\{C^i_*\}_{i\geq 0 }$ over $R$, together with chain map's $f^i: C^i_* \to C^{i+1}_*$, so that we have the following diagram:

\begin{tikzcd}[cells={nodes={minimum height=2em}}]
C^0_* \arrow[r,"f^0"] & 
C^1_* \arrow[r,"f^1"] & 
C^2_* \arrow[r,"f^2"] & 
\cdots
\end{tikzcd}

The filtered simplicial complex defined in \ref{filtered} with inclusion maps for the simplices then becomes a \underline{persistence complex}. To illustrate, below is a portion of a persistence complex with the the chain complexes expanded. 

\begin{tikzcd}[cells={nodes={minimum height=2em}}]
                    \arrow[d,"\partial_3"]&                    \arrow[d,"\partial_3"]&                     \arrow[d,"\partial_3"] \\
C^0_2\arrow[r,"f^0"]\arrow[d,"\partial_2"]& C^1_2\arrow[r,"f^1"]\arrow[d,"\partial_2"]& C^2_2\arrow[r,"f^2"]\arrow[d,"\partial_2"]& \cdots\\
C^0_1\arrow[r,"f^0"]\arrow[d,"\partial_1"]& C^1_1\arrow[r,"f^1"]\arrow[d,"\partial_1"]& C^2_1\arrow[r,"f^2"]\arrow[d,"\partial_1"]& \cdots\\
C^0_0\arrow[r,"f^0"] & C^1_0 \arrow[r,"f^1"] & C^2_0 \arrow[r,"f^2"]& \cdots
\end{tikzcd}
\end{defn}

\begin{defn}[Persistence modules]
A \underline{persistence module} $\mathscr{M}$ is a family of $R$-modules $M^i$, together with homomorphisms $\phi^i: M^i \to M^{i+1}$.
\end{defn}

\begin{thm}[Classification Theorem]
    For a finite persistence module $\mathscr{M}$ with coefficients in the field $F$, 
    \begin{equation}
        H_*(\mathscr{M}; F) \cong \underbrace{\bigoplus_i x^{t_i}F(x)}_\text{free module} \oplus  \underbrace{\left(\bigoplus_j x^{r_j}(F(x)/(x^{s_j}F[x]))\right)}_\text{torsion module}
    \end{equation}
    
    There exists a bijection between the set of free elements and the set of homology generators with birth at $t_i$ and persist for all future parameter values. 
    
    There exists a bijection between the set of torsion elements and the set of homology generators with birth at $r_j$ and death at $r_j + s_j$. 
    
    The classification theorem gives the fundamental characterization of persistence barcode.
\end{thm}

\begin{thm}[Barcode as the persistence analogue of Betti number]
The rank of $H_k^{i\to j}(\mathscr{C}; F)$ gives the number of intervals in the barcode of $H_k^{i\to j}(\mathscr{C}; F)$ spanning the parameter interval $[i,j]$. In particular, $H_*^{i\to j}(\mathscr{C}^i_*; F)$ gives the number of intervals that contain $i$.
\begin{rmk}
As with Betti number, the barcode for $H_k$ does not give the actual structure of the homology group, but just a continuously parameterized rank. The barcode is useful in that it can qualitatively filter out topological noise (since they are ``short-lived" features) and capture significant topological features (features that persist over increasing values of $\epsilon$).
\end{rmk}
\end{thm}


% \listoftables 
% \label{lst:tabs}

% \listoffigures
% \label{lst:figs}